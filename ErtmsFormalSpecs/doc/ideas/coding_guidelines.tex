\documentclass[a4paper, oneside]{scrreprt}
\usepackage[american]{babel}
\usepackage{enumitem}
\usepackage[nohyphen]{underscore} %allow for simple _ instead of \_
\usepackage{xcolor}
\usepackage{xspace}
\usepackage[tablegrid]{vhistory}
\usepackage{hyperref}
\usepackage{scrpage2}

% BEGIN title + general-stuff
\newcommand{\docTitle}{ERTMS Formal Specs Workbench}
\newcommand{\docSubTitle}{Coding Guidelines}

\hypersetup{%
pdftitle = {\docTitle, \docSubTitle},
pdfkeywords = {\docTitle, Version \vhCurrentVersion from \vhCurrentDate},
pdfauthor = {\vhAllAuthorsSet},
colorlinks = true,
linkcolor = black,
citecolor = black,
filecolor = black,
urlcolor = black
}

\pagestyle{scrheadings}
\chead[]{}
\ifoot[\docTitle, \docSubTitle\ -- Version \vhCurrentVersion]{\docTitle, \docSubTitle\ -- Version \vhCurrentVersion}
\cfoot[]{}
\ofoot[\thepage]{\thepage}
% END title + general stuff


% BEGIN specific commands for this file
\makeatletter
% makes the enumeration item label start with the number
% of the current section/subsection/etc.
\newcommand{\thecurrentlevel}{%
  \ifnum\c@section=0\thechapter\else
  \ifnum\c@subsection=0\thesection\else
  \ifnum\c@subsubsection=0\thesubsection\else
  \thesubsubsection\fi\fi\fi}
\makeatother

\setcounter{secnumdepth}{3}
\setcounter{tocdepth}{3}

\setenumerate[1]{leftmargin=*, label=\thecurrentlevel.\arabic*.}
\setenumerate[2]{label*=\arabic*.}
\setenumerate[3]{label*=\arabic*.}
\let\emph\textsl

\newcommand{\code}[1]{\texttt{#1}}
\newcommand{\TODO}{\colorbox{red}{\textcolor{white}{\textbf{\textsf{TODO}}}}\xspace}
\newcommand{\TBD}{\colorbox{green}{\textcolor{white}{\textbf{\textsf{TBD}}}}\xspace}
\newcommand{\remark}[1]{\mbox{}\newline\colorbox{yellow}{\textsf{\textbf{remark:} #1}}\xspace}
\newcommand{\ruleauthor}[2]{\mbox{}\newline\mbox{}\hfill{\footnotesize\textcolor{gray}{\emph{#1, #2}}}\xspace}
% END specific commands for this file


%BEGIN abbreviations for author names
\newcommand{\MD}{Moritz Dorka}
% @ALL:
% extend as needed
% then a \vhEntry as shown below, that's it!

%END abbreviations for author names



% BEGIN OF THE ACTUAL DOCUMENT

\begin{document}
\title{\docTitle}
\subtitle{\docSubTitle, Version: \vhCurrentVersion}
\date{\vhCurrentDate}
\author{\vhListAllAuthorsLongWithAbbrev}
\maketitle


\begin{versionhistory}
\vhEntry{0.1}{31/10/2013}{MD}{initial draft}
% @ALL:
% add your revision information here
\end{versionhistory}

\newpage
\tableofcontents
\newpage

\addchap{How to read this file}
\TODO

\chapter{Domain specific language}

\section{General}
\begin{enumerate}
\item \TODO \code{EMPTY} vs. \code{NOT AVAILABLE()} vs. \code{$[\,]$} \ruleauthor{MD}{31/10/2013}
\item \code{SUM \emph{List} USING X} shall be preferred over \code{REDUCE \emph{List} USING X + RESULT}. \ruleauthor{MD}{31/10/2013}
\item \code{THERE_IS_IN \emph{List} | \emph{Condition}} shall be preferred over \code{MAP \emph{List} | \emph{Condition} USING X}. \ruleauthor{MD}{31/10/2013}
\end{enumerate}


\chapter{Model}

\section{Functions}
\subsection{Cases}

\subsubsection{General}
\begin{enumerate}
\item \label{rule:functions_cases_general_trivialexpressions}\TBD Cases shall only contain trivial expressions (i.e. all logic shall be defined inside pre-conditions). \ruleauthor{MD}{31/10/2013}
\item \TBD Complex expressions and pre-conditions shall not be used together. \remark{this rule is mutually exclusive to \ref{rule:functions_cases_general_trivialexpressions}} \ruleauthor{MD}{31/10/2013}
\end{enumerate}

\subsubsection{Pre-conditions}
\begin{enumerate}
\item Cases with no pre-condition assigned shall always come as the very last case. \ruleauthor{MD}{31/10/2013}
\item Cases with no pre-condition assigned shall always be named "Otherwise". \ruleauthor{MD}{31/10/2013}
\end{enumerate}

\chapter{Tests}

\section{Structure}
\begin{enumerate}
\item The first sub-step of the first step of a test frame shall only contain the expression \code{Kernel.InitializeTestEnvironment()}. \ruleauthor{MD}{31/10/2013}
\begin{enumerate}
\item There shall be no Expectation associated with this sub-step. \ruleauthor{MD}{31/10/2013}
\end{enumerate}
\end{enumerate}


\section{Naming conventions}
\begin{enumerate}
\item Individual test steps shall be named \code{Step \emph{n} - \emph{explanation}}.\\ \emph{Explanation} is a user-defined text to summarize the actions and expectations of the current step. \ruleauthor{MD}{31/10/2013}
\end{enumerate}



\chapter{Git}
\section{General}
\begin{enumerate}
\item Own edits shall always be committed into a personal fork of the EFS repository and then transferred by issuing a pull-request. \ruleauthor{MD}{31/10/2013}
\end{enumerate}

\section{Commits}
\subsection{General}
\begin{enumerate}
\item numerous small commits dealing with a single issue are preferable over few huge commits possibly dealing with many different issues. \ruleauthor{MD}{31/10/2013}
\end{enumerate}

\subsection{commit description texts}
\begin{enumerate}
\item The text shall not contain line-breaks (i.e. consist of a single line only). \ruleauthor{MD}{31/10/2013}
\item Commits dealing with the modeling part of the tool shall begin with \emph{EA_MODEL} followed by a reference to the sections of the subset the commit is related to and a descriptive text. \ruleauthor{MD}{31/10/2013}
\item Commits dealing with the testing part of the tool shall begin with \emph{EA_TEST} followed by a reference to the sections of the subset the commit is related to and a descriptive text. \ruleauthor{MD}{31/10/2013}
\end{enumerate}

\end{document}